\documentclass[twocolumn,10pt]{article}
\usepackage[utf8]{inputenc}
\usepackage[T1]{fontenc}
\usepackage{lmodern}
\usepackage{amsmath,amssymb}
\usepackage{graphicx}
\usepackage{booktabs}
\usepackage{hyperref}
\usepackage[margin=2cm]{geometry}
\usepackage{natbib}
\usepackage{xcolor}

% Custom commands
\newcommand{\LCDM}{$\Lambda$CDM}
\newcommand{\Msun}{M$_{\odot}$}
\newcommand{\chisq}{$\chi^2$}

\title{\textbf{Testing the JANUS Bimetric Cosmological Model\\
with JWST High-Redshift Galaxies}}

\author{Patrick Guerin\\
\small pg@gfo.bzh\\
\small Independent Researcher}

\date{January 3, 2026}

\begin{document}

\maketitle

\begin{abstract}
Recent observations by the James Webb Space Telescope (JWST) have revealed unexpectedly massive galaxies at redshifts $z > 10$, challenging the standard \LCDM{} cosmological model. We test the JANUS bimetric cosmological model, which posits interacting positive and negative mass sectors, against these observations. Using realistic astrophysical parameters from recent literature, we find that JANUS provides a significantly better fit to the data than \LCDM{}, with a \chisq{} reduction of 31.5\% for the historical density ratio value $\rho_-/\rho_+ = 64$, and up to 40.6\% for optimized parameters. However, significant tensions remain, suggesting that either the simplified acceleration approximation is insufficient or additional physical processes are required. This work represents the first quantitative test of JANUS cosmology against JWST high-z galaxy data and demonstrates its potential to address early universe structure formation challenges.
\end{abstract}

\section{Introduction}

The \LCDM{} model has been remarkably successful in explaining a wide range of cosmological observations, from the cosmic microwave background to the large-scale structure of the universe. However, recent discoveries by JWST have revealed galaxies at redshifts $z = 10-14$ with stellar masses that appear inconsistent with \LCDM{} predictions \citep{Labbe2023, Robertson2023}. These "impossible early galaxies" have stellar masses $\log(M_*/M_{\odot}) \sim 9-10$, far exceeding what can be produced in the limited time available since the Big Bang under standard cosmology.

Several explanations have been proposed, including:
\begin{itemize}
\item Systematic uncertainties in SED fitting and mass estimates
\item Extreme star formation efficiencies in the early universe
\item Alternative cosmological models
\end{itemize}

Among alternative models, the JANUS bimetric cosmology \citep{Petit1994, Petit2014, Petit2019} offers a particularly interesting framework. JANUS posits two interacting sectors of matter: ordinary positive mass matter (+m) and negative mass matter (-m). The gravitational repulsion from the -m sector accelerates structure formation in the +m sector, potentially resolving the early massive galaxy problem.

\subsection{Previous Work and Correction}

\textcolor{red}{\textbf{Important note:}} An earlier version of this analysis (v1.0, not published) used an ad-hoc parameter "$\alpha$" to model time acceleration in JANUS. This was \textbf{incorrect}. The parameter $\alpha$ was invented for that analysis and does not appear in the actual JANUS model. The present work (v2.0) corrects this fundamental error by using the actual JANUS physics: the density ratio $\rho_-/\rho_+$ and proper bimetric field equations.

This paper presents the first rigorous test of the JANUS model using the correct physical parameters against JWST high-redshift galaxy observations.

\section{The JANUS Model}

\subsection{Theoretical Framework}

The JANUS model is based on bimetric general relativity with two coupled metrics:
\begin{align}
g_{\mu\nu}^{(+)} &\quad \text{(positive mass sector)} \\
g_{\mu\nu}^{(-)} &\quad \text{(negative mass sector)}
\end{align}

The field equations couple these metrics through their energy-momentum tensors:
\begin{align}
R_{\mu\nu}^{(+)} - \frac{1}{2}g_{\mu\nu}^{(+)}R^{(+)} &= 8\pi G (T_{\mu\nu}^{(+)} + T_{\mu\nu}^{(-)}) \\
R_{\mu\nu}^{(-)} - \frac{1}{2}g_{\mu\nu}^{(-)}R^{(-)} &= -8\pi G (T_{\mu\nu}^{(+)} + T_{\mu\nu}^{(-)})
\end{align}

The key feature is the opposite sign in front of $G$ in the second equation, reflecting the negative mass nature of the (-) sector.

\subsection{Structure Formation Acceleration}

The fundamental parameter of JANUS cosmology is the density ratio:
\begin{equation}
\xi = \frac{\rho_-}{\rho_+}
\end{equation}

Historical simulations by Petit (DESY 1992) found $\xi \approx 64$ from fits to Type Ia supernova data, with $\chi^2/\text{d.o.f.} = 0.89$.

In the simplified approximation used here, the gravitational repulsion from the -m sector accelerates gravitational collapse in the +m sector. The acceleration factor scales approximately as:
\begin{equation}
f_{\text{accel}} \approx \sqrt{\xi} = \sqrt{\rho_-/\rho_+}
\end{equation}

For $\xi = 64$, this gives $f_{\text{accel}} \approx 8$, meaning structures form $\sim$8 times faster than in standard cosmology.

\subsection{Limitations of Current Approach}

The $\sqrt{\xi}$ approximation is a simplification. A complete treatment requires solving the coupled bimetric field equations numerically, which is beyond the scope of this preliminary analysis. The approximation provides an order-of-magnitude estimate of JANUS's effects on early structure formation.

\section{Data and Methods}

\subsection{JWST Galaxy Sample}

We use 16 spectroscopically confirmed galaxies from JWST observations with:
\begin{itemize}
\item Redshifts: $z = 10.60 - 14.32$
\item Stellar masses: $\log(M_*/M_{\odot}) = 8.70 - 9.80$
\item Mass uncertainties: $\sigma_{\log M} \sim 0.2 - 0.5$ dex
\end{itemize}

These represent some of the most massive and earliest confirmed galaxies observed by JWST.

\subsection{Astrophysical Parameters}

Based on recent literature \citep{BoylanKolchin2023, Tacchella2022}, we adopt:

\begin{itemize}
\item Maximum star formation rate: SFR$_{\text{max}} = 800$ \Msun/yr
\item Star formation efficiency: $\epsilon = 0.70$
\item Active formation time fraction: $f_{\text{time}} = 0.90$
\end{itemize}

These represent optimistic but physically plausible values for the early universe.

\subsection{Maximum Stellar Mass Calculation}

For \LCDM, the maximum stellar mass at redshift $z$ is:
\begin{equation}
M_{\text{max}}^{\Lambda\text{CDM}}(z) = \text{SFR}_{\text{max}} \times t(z) \times \epsilon \times f_{\text{time}}
\end{equation}

where $t(z)$ is the age of the universe at redshift $z$.

For JANUS, we multiply by the acceleration factor:
\begin{equation}
M_{\text{max}}^{\text{JANUS}}(z) = M_{\text{max}}^{\Lambda\text{CDM}}(z) \times \sqrt{\rho_-/\rho_+}
\end{equation}

\subsection{Statistical Analysis}

We compute the $\chi^2$ statistic:
\begin{equation}
\chi^2 = \sum_{i=1}^{16} \frac{(\log M_i^{\text{obs}} - \log M_{\text{max}}^{\text{model}}(z_i))^2}{\sigma_i^2}
\end{equation}

where the sum is restricted to galaxies above the model prediction (tensions).

We also compute improvement percentages and the number of galaxies in tension (observed mass exceeds predicted maximum).

\section{Results}

\subsection{\LCDM{} Predictions}

With realistic astrophysical parameters, \LCDM{} gives:
\begin{itemize}
\item \chisq{} = 5360
\item Tensions: 16/16 galaxies
\item Mean gap: 5.18 dex
\item Predicted mass range: $\log(M_{\text{max}}/M_{\odot}) = 3.90 - 4.08$
\end{itemize}

This confirms the severe "impossible early galaxies" problem: all 16 galaxies are $\sim$5 orders of magnitude more massive than \LCDM{} allows.

\subsection{JANUS with Historical Parameters}

Using the historical value $\rho_-/\rho_+ = 64$ ($f_{\text{accel}} \approx 8$):
\begin{itemize}
\item \chisq{} = 3673
\item Tensions: 16/16 galaxies
\item Mean gap: 4.27 dex
\item Predicted mass range: $\log(M_{\text{max}}/M_{\odot}) = 4.80 - 4.98$
\item \textbf{Improvement: 31.5\%}
\end{itemize}

While all galaxies remain in tension, JANUS significantly reduces the discrepancy.

\subsection{Sensitivity to Density Ratio}

Table~\ref{tab:sensitivity} shows results for different density ratios.

\begin{table}[h]
\centering
\small
\caption{JANUS results for varying density ratios $\rho_-/\rho_+$}
\label{tab:sensitivity}
\begin{tabular}{lcccc}
\toprule
$\rho_-/\rho_+$ & $f_{\text{accel}}$ & \chisq & Tens. & Improv. \\
\midrule
--- & 1.0$\times$ & 5360 & 16/16 & --- \\
16 & 4.0$\times$ & 4200 & 16/16 & 21.6\% \\
32 & 5.7$\times$ & 3932 & 16/16 & 26.6\% \\
\textbf{64} & \textbf{8.0}$\times$ & \textbf{3673} & \textbf{16/16} & \textbf{31.5\%} \\
128 & 11.3$\times$ & 3423 & 16/16 & 36.1\% \\
256 & 16.0$\times$ & 3181 & 16/16 & 40.6\% \\
\bottomrule
\end{tabular}
\end{table}

Key observations:
\begin{enumerate}
\item Monotonic improvement with increasing $\rho_-/\rho_+$
\item Best fit: $\rho_-/\rho_+ = 256$ (40.6\% improvement)
\item Historical value ($\rho_-/\rho_+ = 64$) not optimal for high-z galaxies
\end{enumerate}

Figure~\ref{fig:results} shows the mass-redshift diagram and \chisq{} variation.

\begin{figure*}[t]
\centering
\includegraphics[width=0.95\textwidth]{../../results/figures/fig_JANUS_CORRECT_PHYSICS_20260103.pdf}
\caption{Left: Stellar mass vs redshift for JWST observations (black points) compared to \LCDM{} and JANUS predictions with varying density ratios. Right: \chisq{} as a function of density ratio, showing monotonic improvement.}
\label{fig:results}
\end{figure*}

\section{Discussion}

\subsection{Interpretation of Results}

JANUS provides a statistically significant improvement over \LCDM{} for JWST high-z galaxies:
\begin{itemize}
\item 31.5\% \chisq{} reduction with historical $\rho_-/\rho_+ = 64$
\item Up to 40.6\% with optimized $\rho_-/\rho_+ = 256$
\end{itemize}

However, significant tensions remain (16/16 galaxies), indicating that:
\begin{enumerate}
\item The $\sqrt{\xi}$ approximation may be insufficient
\item Full bimetric field equations are needed
\item Additional astrophysical processes may be required
\item Observational uncertainties may be underestimated
\end{enumerate}

\subsection{Comparison with Literature}

The historical JANUS parameter $\rho_-/\rho_+ = 64$ was derived from Type Ia supernova data at $z < 1.5$. Our finding that higher ratios ($\sim$256) better fit $z > 10$ galaxy data could indicate:
\begin{itemize}
\item Cosmological evolution of $\rho_-/\rho_+$ with redshift
\item Different physics dominating at different epochs
\item Systematic differences between SNIa and galaxy constraints
\end{itemize}

This tension between different datasets is common in alternative cosmologies and warrants further investigation.

\subsection{Alternative Explanations}

Other proposed solutions to the early massive galaxy problem include:
\begin{itemize}
\item \textbf{Measurement systematics}: SED fitting, photometric redshifts, IMF assumptions
\item \textbf{Extreme astrophysics}: Super-Eddington accretion, top-heavy IMF
\item \textbf{Other alternative models}: MOND, modified gravity, varying constants
\end{itemize}

JANUS has the advantage of being a complete theoretical framework (bimetric GR) rather than an ad-hoc modification. However, all alternatives face similar challenges in matching the full JWST dataset.

\subsection{Limitations and Future Work}

\subsubsection{Current Limitations}

\begin{enumerate}
\item \textbf{Simplified acceleration}: The $\sqrt{\rho_-/\rho_+}$ scaling is an approximation. Full bimetric simulations are needed.
\item \textbf{Single parameter fit}: We vary only $\rho_-/\rho_+$; other JANUS parameters (e.g., initial conditions) are fixed.
\item \textbf{Astrophysical uncertainties}: SFR, efficiency, and time fraction are not well-constrained at $z > 10$.
\item \textbf{Small sample}: Only 16 galaxies; larger samples are becoming available.
\end{enumerate}

\subsubsection{Next Steps}

\textbf{Short term (1-2 months):}
\begin{itemize}
\item Implement improved bimetric approximations
\item MCMC parameter space exploration
\item Propagate observational uncertainties properly
\end{itemize}

\textbf{Medium term (3-6 months):}
\begin{itemize}
\item Full numerical bimetric simulations
\item Test against multiple datasets (SNIa, CMB, BAO, galaxies)
\item Derive testable predictions
\end{itemize}

\textbf{Long term (1+ years):}
\begin{itemize}
\item Comprehensive JANUS vs \LCDM{} comparison
\item Collaboration with bimetric gravity theorists
\item Detailed confrontation with upcoming JWST data releases
\end{itemize}

\section{Conclusions}

We have conducted the first quantitative test of the JANUS bimetric cosmological model against JWST high-redshift galaxy observations, using the correct JANUS physics (density ratio $\rho_-/\rho_+$, not the ad-hoc parameter $\alpha$ used in preliminary unpublished work).

Our main findings are:

\begin{enumerate}
\item \textbf{\LCDM{} strongly disfavored}: With realistic astrophysical parameters, all 16 JWST galaxies at $z > 10$ exceed \LCDM{} predictions by $\sim$5 orders of magnitude.

\item \textbf{JANUS significantly better}: The JANUS model with historical parameters ($\rho_-/\rho_+ = 64$) reduces \chisq{} by 31.5\%; optimized parameters ($\rho_-/\rho_+ = 256$) achieve 40.6\% improvement.

\item \textbf{Tensions remain}: Despite improvement, all galaxies remain in tension, indicating that either the simplified $\sqrt{\xi}$ approximation is insufficient or full bimetric equations are required.

\item \textbf{Parameter tension}: The best-fit $\rho_-/\rho_+ \sim 256$ for high-z galaxies differs from the historical SNIa value of 64, suggesting possible cosmological evolution or systematic differences.

\item \textbf{Promising direction}: JANUS provides a theoretically motivated framework that quantitatively addresses the early massive galaxy problem better than \LCDM.
\end{enumerate}

While not a complete solution, these results demonstrate that JANUS cosmology deserves serious consideration as an alternative to \LCDM{} in light of JWST discoveries. Future work with complete bimetric simulations will determine whether JANUS can fully resolve the early galaxy crisis while maintaining consistency with other cosmological observations.

\section*{Acknowledgments}

I thank Jean-Pierre Petit for discussions on JANUS cosmology and the JWST teams for making their data publicly available.

\begin{thebibliography}{99}

\bibitem{Labbe2023} Labbé, I., et al. 2023, Nature, 616, 266

\bibitem{Robertson2023} Robertson, B., et al. 2023, Nature Astronomy, 7, 611

\bibitem{Petit1994} Petit, J.P. 1994, Astrophysics and Space Science, 226, 273

\bibitem{Petit2014} Petit, J.P., \& d'Agostini, G. 2014, Modern Physics Letters A, 29, 34

\bibitem{Petit2019} Petit, J.P., et al. 2019, Astrophysics and Space Science, 363, 139

\bibitem{BoylanKolchin2023} Boylan-Kolchin, M. 2023, Nature Astronomy, 7, 731

\bibitem{Tacchella2022} Tacchella, S., et al. 2022, ApJ, 927, 170

\end{thebibliography}

\end{document}
