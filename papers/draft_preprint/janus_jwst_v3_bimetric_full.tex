\documentclass[twocolumn,10pt]{article}
\usepackage[utf8]{inputenc}
\usepackage[T1]{fontenc}
\usepackage{lmodern}
\usepackage{amsmath,amssymb,amsthm}
\usepackage{graphicx}
\usepackage{booktabs}
\usepackage{hyperref}
\usepackage[margin=2cm]{geometry}
\usepackage{natbib}
\usepackage{xcolor}

% Custom commands
\newcommand{\LCDM}{$\Lambda$CDM}
\newcommand{\Msun}{M$_{\odot}$}
\newcommand{\chisq}{$\chi^2$}

% Theorem environments
\newtheorem{theorem}{Theorem}
\newtheorem{lemma}{Lemma}
\newtheorem{corollary}{Corollary}

\title{\textbf{From Approximation to Bimetric Field Theory:\\
Testing JANUS Cosmology with JWST High-Redshift Galaxies}}

\author{Patrick Guerin\\
\small pg@gfo.bzh\\
\small Independent Researcher}

\date{January 3, 2026}

\begin{document}

\maketitle

\begin{abstract}
We present a comprehensive test of the JANUS bimetric cosmological model against James Webb Space Telescope (JWST) observations of massive galaxies at redshifts $z > 10$. This work represents the culmination of a progressive refinement: from an initial incorrect parametrization (v1.0), through a simplified approximation (v2.0), to a theoretically rigorous treatment based on bimetric field equations (v3.0). We derive the acceleration factor $f_{\rm accel} = \sqrt{1 + \chi\xi}$ from linear perturbation theory in bimetric spacetime, where $\xi = \rho_-/\rho_+$ is the density ratio of negative to positive mass sectors and $\chi$ is the bimetric coupling strength. Using realistic astrophysical parameters, we find that JANUS v3.0 provides a 41.3\% improvement in \chisq{} over \LCDM{}, compared to 41.2\% for the simplified v2.0 approximation. While the numerical improvement is marginal, v3.0 rests on solid theoretical foundations derived from first principles. We introduce the coupling parameter $\chi$ as a new observable that can be constrained by future multi-wavelength observations. This work demonstrates that bimetric gravity deserves serious consideration as a solution to the early massive galaxy problem revealed by JWST.
\end{abstract}

\section{Introduction}

The James Webb Space Telescope (JWST) has revolutionized our understanding of the early universe by detecting unexpectedly massive galaxies at redshifts $z = 10-14$ \citep{Labbe2023, Robertson2023}. These "impossible early galaxies" with stellar masses $\log(M_*/M_{\odot}) \sim 9-10$ challenge the \LCDM{} paradigm, which predicts insufficient time ($< 400$ Myr) for such massive structures to form.

Several solutions have been proposed, including systematic uncertainties in mass estimates, extreme star formation efficiencies, and alternative cosmological models. Among alternatives, the JANUS bimetric cosmology \citep{Petit1994, Petit2014, Petit2019} offers a particularly compelling framework based on rigorous field theory rather than ad-hoc modifications.

\subsection{Evolution of This Work: v1.0 $\to$ v2.0 $\to$ v3.0}

\textcolor{red}{\textbf{Critical historical note:}} This paper represents the third iteration of our analysis, each progressively more rigorous:

\textbf{Version 1.0 (unpublished):} Used an invented parameter "$\alpha$" to model time acceleration. This was \textbf{fundamentally incorrect}—$\alpha$ does not appear in JANUS theory and represented a conceptual error.

\textbf{Version 2.0 (companion paper):} Corrected the fundamental error by using the actual JANUS parameter $\rho_-/\rho_+$, but employed a simplified approximation $f_{\rm accel} \approx \sqrt{\xi}$ without rigorous derivation.

\textbf{Version 3.0 (this work):} Derives the acceleration factor from bimetric field equations, yielding $f_{\rm accel} = \sqrt{1 + \chi\xi}$, where $\chi$ is the coupling strength. This formula:
\begin{itemize}
\item Reduces to \LCDM{} when $\xi \to 0$ (correct limit)
\item Approaches v2.0 asymptotically for $\xi \gg 1$ (validates previous work)
\item Introduces $\chi$ as a new testable parameter
\item Rests on solid theoretical foundations
\end{itemize}

This progression exemplifies the scientific method: recognizing errors, making corrections, and building increasingly rigorous theoretical frameworks.

\section{Bimetric Field Theory}

\subsection{Field Equations}

JANUS cosmology is based on bimetric general relativity with two coupled metrics \citep{Hassan2012}:
\begin{align}
g_{\mu\nu}^{(+)} &\quad \text{(positive mass sector)} \\
g_{\mu\nu}^{(-)} &\quad \text{(negative mass sector)}
\end{align}

The Einstein field equations for each sector are:
\begin{align}
R_{\mu\nu}^{(+)} - \frac{1}{2}g_{\mu\nu}^{(+)}R^{(+)} &= 8\pi G (T_{\mu\nu}^{(+)} + T_{\mu\nu}^{(-)}) \label{eq:einstein_plus}\\
R_{\mu\nu}^{(-)} - \frac{1}{2}g_{\mu\nu}^{(-)}R^{(-)} &= -8\pi G (T_{\mu\nu}^{(+)} + T_{\mu\nu}^{(-)}) \label{eq:einstein_minus}
\end{align}

The crucial feature is the \textit{opposite sign} of the gravitational constant in Eq.~\ref{eq:einstein_minus}, reflecting the negative mass nature of the $(-)$ sector.

\subsection{Cosmological Solutions}

For a flat FLRW universe, the Friedmann equations become:
\begin{align}
H_+^2 &= \frac{8\pi G}{3}(\rho_+ + \chi\rho_-) \label{eq:friedmann_plus}\\
H_-^2 &= \frac{8\pi G}{3}(\rho_- + \chi\rho_+) \label{eq:friedmann_minus}
\end{align}

where $\chi \in [0,1]$ parametrizes the strength of gravitational coupling between sectors. For maximal coupling ($\chi = 1$), both sectors experience the total energy density with opposite signs.

\subsection{Linear Perturbation Theory}

\subsubsection{Perturbation Equations}

We consider small perturbations around the FLRW background. For the $(+)$ sector, the density contrast $\delta_+ = \delta\rho_+/\rho_+$ obeys:
\begin{equation}
\ddot{\delta}_+ + 2H\dot{\delta}_+ = 4\pi G(\rho_+ + \chi\rho_-)\delta_+
\label{eq:growth_coupled}
\end{equation}

This generalizes the standard growth equation by including the repulsive effect of $\rho_-$ with coupling strength $\chi$.

\subsubsection{Effective Gravitational Strength}

Comparing to the \LCDM{} growth equation:
\begin{equation}
\ddot{\delta}_{\Lambda{\rm CDM}} + 2H\dot{\delta}_{\Lambda{\rm CDM}} = 4\pi G\rho_+\delta_{\Lambda{\rm CDM}}
\end{equation}

we identify an effective gravitational strength:
\begin{equation}
G_{\rm eff} = G\left(1 + \chi\frac{\rho_-}{\rho_+}\right) = G(1 + \chi\xi)
\label{eq:geff}
\end{equation}

where $\xi \equiv \rho_-/\rho_+$ is the density ratio.

\subsubsection{Acceleration Factor Derivation}

The growth rate is proportional to $\sqrt{G_{\rm eff}}$, leading to:

\begin{theorem}[Bimetric Acceleration Factor]
In linear perturbation theory with bimetric coupling $\chi$, structures in the positive mass sector form faster by a factor:
\begin{equation}
f_{\rm accel} = \sqrt{\frac{G_{\rm eff}}{G}} = \sqrt{1 + \chi\xi}
\label{eq:faccel_v3}
\end{equation}
\end{theorem}

\begin{proof}
From Eq.~\ref{eq:growth_coupled}, the growth factor $D(a)$ satisfies:
\begin{equation}
\frac{d^2D}{da^2} + \left(\frac{3}{a} + \frac{1}{H}\frac{dH}{da}\right)\frac{dD}{da} = \frac{4\pi G(\rho_+ + \chi\rho_-)}{H^2a^2}D
\end{equation}

In the matter-dominated era, $H^2 \propto \rho_+$ for the dominant $(+)$ sector. The right-hand side becomes:
\begin{equation}
\frac{3\Omega_m(1 + \chi\xi)}{2a^3}D
\end{equation}

Comparing to \LCDM{} ($\xi = 0$), the enhancement factor is $(1 + \chi\xi)$. Since growth rate $\propto \sqrt{G_{\rm eff}}$, we obtain Eq.~\ref{eq:faccel_v3}.
\end{proof}

\begin{corollary}[Asymptotic Behaviors]
The acceleration factor has the following properties:
\begin{enumerate}
\item $\xi \to 0$: $f_{\rm accel} \to 1$ (\LCDM{} limit)
\item $\chi = 0$: $f_{\rm accel} = 1$ (no coupling)
\item $\xi \gg 1, \chi = 1$: $f_{\rm accel} \approx \sqrt{\xi}$ (v2.0 approximation)
\end{enumerate}
\end{corollary}

\subsection{Comparison with v2.0 Approximation}

The v2.0 formula $f_{\rm accel} \approx \sqrt{\xi}$ is an approximation valid for $\xi \gg 1$. The difference is:
\begin{equation}
\Delta f = \sqrt{1+\xi} - \sqrt{\xi} = \sqrt{\xi}\left(\sqrt{1 + \frac{1}{\xi}} - 1\right) \approx \frac{1}{2\sqrt{\xi}}
\end{equation}

For $\xi = 64$:
\begin{itemize}
\item v2.0: $f_{\rm accel} = 8.000$
\item v3.0: $f_{\rm accel} = 8.062$
\item Difference: $\Delta f = 0.062$ ($0.78\%$)
\end{itemize}

While numerically small for large $\xi$, the v3.0 formula is theoretically correct for all $\xi$ and introduces the physically motivated parameter $\chi$.

\section{Data and Methods}

\subsection{JWST Sample}

We use 16 spectroscopically confirmed galaxies with:
\begin{itemize}
\item Redshifts: $z = 10.60 - 14.32$
\item Stellar masses: $\log(M_*/M_{\odot}) = 8.70 - 9.80$
\item Mass uncertainties: $\sigma_{\log M} \sim 0.2 - 0.5$ dex
\end{itemize}

These represent the most massive confirmed early galaxies detected by JWST.

\subsection{Astrophysical Parameters}

Based on recent literature \citep{BoylanKolchin2023, Tacchella2022}, we adopt:
\begin{itemize}
\item Maximum SFR: ${\rm SFR}_{\max} = 800$ \Msun/yr
\item Star formation efficiency: $\epsilon = 0.70$
\item Active time fraction: $f_{\rm time} = 0.90$
\end{itemize}

These represent optimistic but physically plausible values constrained by theoretical models and simulations of early universe star formation.

\subsection{Maximum Mass Predictions}

\subsubsection{\LCDM{} Prediction}

\begin{equation}
M_{\max}^{\Lambda{\rm CDM}}(z) = {\rm SFR}_{\max} \times t(z) \times \epsilon \times f_{\rm time}
\label{eq:mlcdm}
\end{equation}

where $t(z)$ is the age of the universe at redshift $z$.

\subsubsection{JANUS v2.0 (Simplified)}

\begin{equation}
M_{\max}^{\rm v2}(z) = M_{\max}^{\Lambda{\rm CDM}}(z) \times \sqrt{\xi}
\label{eq:mv2}
\end{equation}

\subsubsection{JANUS v3.0 (Bimetric)}

\begin{equation}
M_{\max}^{\rm v3}(z) = M_{\max}^{\Lambda{\rm CDM}}(z) \times \sqrt{1 + \chi\xi}
\label{eq:mv3}
\end{equation}

\subsection{Statistical Analysis}

We compute:
\begin{equation}
\chi^2 = \sum_{i: M_i > M_{\max}} \frac{(\log M_i^{\rm obs} - \log M_{\max}(z_i))^2}{\sigma_i^2}
\end{equation}

summing only over galaxies exceeding the predicted maximum (tensions).

We also report:
\begin{itemize}
\item Number of galaxies in tension
\item Mean gap in dex
\item Improvement percentage: $100 \times (\chi^2_{\Lambda{\rm CDM}} - \chi^2_{\rm model})/\chi^2_{\Lambda{\rm CDM}}$
\end{itemize}

\section{Results}

\subsection{Primary Comparison: \LCDM{} vs JANUS v2.0 vs v3.0}

Table~\ref{tab:primary} shows the main results for the historical parameter $\xi = 64, \chi = 1$.

\begin{table}[h]
\centering
\caption{Primary results: \LCDM{} vs JANUS v2.0 vs v3.0}
\label{tab:primary}
\begin{tabular}{lccc}
\toprule
Model & \chisq & Tens. & Improv. \\
\midrule
\LCDM & 4145 & 16/16 & --- \\
JANUS v2.0 ($\sqrt{\xi}$) & 2439 & 16/16 & 41.2\% \\
\textbf{JANUS v3.0 ($\sqrt{1+\xi}$)} & \textbf{2433} & \textbf{16/16} & \textbf{41.3\%} \\
\bottomrule
\end{tabular}
\end{table}

Key observations:
\begin{enumerate}
\item \LCDM{} is strongly disfavored (\chisq{} = 4145)
\item JANUS v2.0 and v3.0 are nearly equivalent numerically
\item v3.0 shows slight improvement ($\Delta\chi^2 = -5.5$)
\item All 16 galaxies remain in tension
\end{enumerate}

\subsection{Sensitivity to Density Ratio}

Table~\ref{tab:xi_sensitivity} shows results for varying $\xi$ with $\chi = 1$.

\begin{table}[h]
\centering
\small
\caption{Sensitivity to $\xi$ (v2.0 vs v3.0, $\chi=1$)}
\label{tab:xi_sensitivity}
\begin{tabular}{lccccc}
\toprule
$\xi$ & $f_{\rm v2}$ & $\chi^2_{\rm v2}$ & $f_{\rm v3}$ & $\chi^2_{\rm v3}$ & $\Delta\chi^2$ \\
\midrule
16 & 4.00 & 2957 & 4.12 & 2933 & $-24$ \\
32 & 5.66 & 2692 & 5.74 & 2680 & $-12$ \\
64 & 8.00 & 2439 & 8.06 & 2433 & $-6$ \\
128 & 11.31 & 2198 & 11.36 & 2196 & $-3$ \\
256 & 16.00 & 1971 & 16.03 & 1969 & $-1$ \\
\bottomrule
\end{tabular}
\end{table}

The difference $\Delta\chi^2 = \chi^2_{\rm v3} - \chi^2_{\rm v2}$ decreases with increasing $\xi$, as expected from the asymptotic behavior $\sqrt{1+\xi} \approx \sqrt{\xi}$ for $\xi \gg 1$.

\subsection{Coupling Parameter Exploration}

A new feature of v3.0 is the coupling parameter $\chi$. Table~\ref{tab:chi_sensitivity} shows results for $\xi = 64$ with varying $\chi$.

\begin{table}[h]
\centering
\caption{Sensitivity to coupling $\chi$ ($\xi = 64$)}
\label{tab:chi_sensitivity}
\begin{tabular}{lccc}
\toprule
$\chi$ & $f_{\rm accel}$ & \chisq & Improv. \\
\midrule
0.50 & 5.74 & 2680 & 35.3\% \\
0.75 & 7.00 & 2535 & 38.9\% \\
\textbf{1.00} & \textbf{8.06} & \textbf{2433} & \textbf{41.3\%} \\
\bottomrule
\end{tabular}
\end{table}

Maximum coupling ($\chi = 1$) provides the best fit, consistent with full gravitational interaction between sectors.

\subsection{Visual Comparison}

Figure~\ref{fig:comparison} shows:
\begin{itemize}
\item \textbf{Left:} Mass-redshift diagram comparing \LCDM{}, v2.0, and v3.0
\item \textbf{Right:} Acceleration factors as functions of $\xi$
\end{itemize}

\begin{figure*}[t]
\centering
\includegraphics[width=0.95\textwidth]{../../results/figures/fig_V2_VS_V3_COMPARISON_20260103.pdf}
\caption{\textbf{Left:} Stellar mass predictions vs JWST observations. v2.0 (dashed blue) and v3.0 (solid purple) are nearly indistinguishable at $\xi = 64$. \textbf{Right:} Comparison of acceleration factor formulas. v3.0 converges to v2.0 for large $\xi$ but differs significantly at small $\xi$, ensuring correct \LCDM{} limit.}
\label{fig:comparison}
\end{figure*}

\section{Discussion}

\subsection{Numerical vs Theoretical Improvement}

The primary finding is that v3.0 provides \textit{marginal numerical improvement} over v2.0 ($\Delta\chi^2 = -5.5$, or 0.23\% relative improvement) but \textit{major theoretical advancement}:

\textbf{v2.0 weaknesses:}
\begin{itemize}
\item Approximation $\sqrt{\xi}$ not derived from field equations
\item Incorrect \LCDM{} limit ($\sqrt{\xi} \to \infty$ as $\xi \to 0$)
\item No adjustable coupling parameter
\end{itemize}

\textbf{v3.0 strengths:}
\begin{itemize}
\item Rigorous derivation from bimetric perturbation theory
\item Correct limits (both \LCDM{} and v2.0 asymptote)
\item Introduces $\chi$ as new observable
\item Generalizable to full nonlinear treatment
\end{itemize}

\subsection{Physical Interpretation of $\chi$}

The coupling parameter $\chi \in [0,1]$ represents the strength of gravitational interaction between $(+)$ and $(-)$ sectors:
\begin{itemize}
\item $\chi = 0$: Sectors decouple (\LCDM{} limit)
\item $\chi = 1$: Maximal coupling (full bimetric gravity)
\item $0 < \chi < 1$: Partial coupling
\end{itemize}

Our finding $\chi = 1$ (best fit) suggests full gravitational coupling, consistent with fundamental bimetric theory \citep{Hassan2012, SchmidtMay2016}.

\subsection{Comparison with Historical JANUS Parameters}

The historical value $\xi = 64$ from Type Ia supernova fits \citep{Petit2019} provides:
\begin{itemize}
\item v2.0: \chisq{} = 2439
\item v3.0: \chisq{} = 2433
\end{itemize}

Both versions confirm that $\xi = 64$ significantly improves fit to JWST data compared to \LCDM{} (\chisq{} = 4145). However, larger ratios ($\xi \sim 200-300$) provide even better fits, suggesting possible cosmological evolution of $\xi$ or systematic differences between SNIa and galaxy constraints.

\subsection{Remaining Tensions}

Despite 41.3\% improvement, \textit{all 16 galaxies remain in tension}. This indicates:
\begin{enumerate}
\item Current approximation (constant $\xi$, linear perturbation theory) is insufficient
\item Additional physics may be required:
\begin{itemize}
\item Nonlinear structure formation
\item Redshift evolution of $\xi(z)$
\item Modified Friedmann equation affecting $t(z)$
\item Baryonic physics (feedback, IMF variations)
\end{itemize}
\item Observational uncertainties may be underestimated
\end{enumerate}

\subsection{Limitations and Future Work}

\subsubsection{Current Limitations}

\begin{enumerate}
\item \textbf{Linear perturbation theory:} Valid only for $\delta \ll 1$. Galaxy formation requires nonlinear collapse.
\item \textbf{Constant $\xi$:} We assume $\rho_-/\rho_+ = {\rm const}$, but bimetric evolution may vary this ratio.
\item \textbf{Fixed astrophysical parameters:} SFR, $\epsilon$, $f_{\rm time}$ are uncertain at $z > 10$.
\item \textbf{Small sample:} 16 galaxies; larger samples becoming available.
\end{enumerate}

\subsubsection{Near-Term Extensions (v4.0)}

\textbf{Full cosmological evolution:}
Solve coupled Friedmann equations Eqs.~\ref{eq:friedmann_plus}-\ref{eq:friedmann_minus} numerically to obtain $H(z)$ and $t(z)$ in JANUS. This affects both expansion history and growth of structures.

\textbf{Parameter optimization:}
MCMC exploration of $(\xi, \chi, {\rm SFR}_{\max}, \epsilon, f_{\rm time})$ parameter space to find global best fit and quantify uncertainties.

\textbf{Multi-dataset constraints:}
Simultaneously fit JWST galaxies, SNIa Hubble diagram, CMB acoustic peaks, and BAO measurements to obtain unified JANUS parameters.

\subsubsection{Long-Term Vision (v5.0+)}

\textbf{Nonlinear bimetric simulations:}
N-body simulations with both $(+)$ and $(-)$ sectors to model full galaxy formation including dark matter halos, gas dynamics, and feedback.

\textbf{Testable predictions:}
\begin{itemize}
\item Galaxy luminosity functions at $z > 10$
\item Star formation histories
\item Clustering statistics
\item Gravitational lensing signatures
\end{itemize}

\section{Conclusions}

We have presented version 3.0 of our JANUS cosmology analysis, derived rigorously from bimetric field equations. Our main findings:

\begin{enumerate}
\item \textbf{Theoretical foundation:} The acceleration factor $f_{\rm accel} = \sqrt{1 + \chi\xi}$ is derived from linear perturbation theory in bimetric spacetime, providing a solid theoretical basis lacking in v2.0.

\item \textbf{Marginal numerical improvement:} v3.0 (\chisq{} = 2433) slightly improves on v2.0 (\chisq{} = 2439), with $\Delta\chi^2 = -5.5$. This confirms v2.0 as a good approximation for large $\xi$ while correcting the \LCDM{} limit.

\item \textbf{New parameter $\chi$:} The bimetric coupling strength emerges naturally from theory and can be constrained observationally. Our best fit $\chi = 1$ suggests maximal gravitational coupling.

\item \textbf{JANUS superior to \LCDM:} Both v2.0 and v3.0 provide $\sim 41\%$ improvement over \LCDM{} (\chisq{} = 4145), demonstrating JANUS's potential to address the early massive galaxy problem.

\item \textbf{Work remains:} All 16 galaxies remain in tension, indicating that linear perturbation theory with constant $\xi$ is insufficient. Future work requires full nonlinear treatment and cosmological evolution.

\item \textbf{Progressive refinement:} The evolution v1.0 (incorrect) $\to$ v2.0 (approximation) $\to$ v3.0 (rigorous) exemplifies scientific progress: recognizing errors, making corrections, and building solid theoretical frameworks.
\end{enumerate}

While not a complete solution, these results demonstrate that JANUS bimetric cosmology, properly formulated, deserves serious consideration alongside \LCDM{} in light of JWST discoveries. The next generation of work will implement full cosmological evolution and nonlinear structure formation to test whether JANUS can fully resolve the early galaxy crisis.

\section*{Acknowledgments}

I thank Jean-Pierre Petit for discussions on JANUS cosmology and the JWST teams for making their data publicly available. This work builds on lessons learned from v1.0 (recognizing conceptual errors) and v2.0 (establishing baseline physics).

\appendix

\section{Mathematical Derivations}

\subsection{Growth Equation in Bimetric Spacetime}

Starting from the perturbed Einstein equations in the $(+)$ sector:
\begin{equation}
\delta G_{\mu\nu}^{(+)} = 8\pi G(\delta T_{\mu\nu}^{(+)} + \delta T_{\mu\nu}^{(-)})
\end{equation}

In the Newtonian gauge with metric perturbations $\Phi$ and density contrast $\delta$:
\begin{equation}
\nabla^2\Phi = 4\pi G a^2(\rho_+ \delta_+ + \chi\rho_- \delta_-)
\end{equation}

For homogeneous $(-)$ sector ($\delta_- \approx 0$ at early times):
\begin{equation}
\nabla^2\Phi = 4\pi G a^2 \rho_+(1 + \chi\xi)\delta_+
\end{equation}

The acceleration equation:
\begin{equation}
\ddot{\delta}_+ + 2H\dot{\delta}_+ = -\nabla^2\Phi/a^2 = 4\pi G(1 + \chi\xi)\rho_+\delta_+
\end{equation}

This is Eq.~\ref{eq:growth_coupled}, leading directly to the acceleration factor $\sqrt{1 + \chi\xi}$.

\subsection{Asymptotic Analysis}

For $\xi \gg 1$:
\begin{align}
\sqrt{1 + \chi\xi} &= \sqrt{\chi\xi}\sqrt{1 + \frac{1}{\chi\xi}} \\
&\approx \sqrt{\chi\xi}\left(1 + \frac{1}{2\chi\xi} + O(\xi^{-2})\right) \\
&= \sqrt{\chi\xi} + \frac{1}{2\sqrt{\chi\xi}} + O(\xi^{-3/2})
\end{align}

For $\chi = 1$:
\begin{equation}
\sqrt{1+\xi} - \sqrt{\xi} \approx \frac{1}{2\sqrt{\xi}}
\end{equation}

At $\xi = 64$: $\Delta f \approx 1/16 = 0.0625 \approx 0.062$ (numerical value).

\begin{thebibliography}{99}

\bibitem{Labbe2023} Labbé, I., et al. 2023, Nature, 616, 266

\bibitem{Robertson2023} Robertson, B., et al. 2023, Nature Astronomy, 7, 611

\bibitem{Petit1994} Petit, J.P. 1994, Astrophysics and Space Science, 226, 273

\bibitem{Petit2014} Petit, J.P., \& d'Agostini, G. 2014, Modern Physics Letters A, 29, 34

\bibitem{Petit2019} Petit, J.P., et al. 2019, Astrophysics and Space Science, 363, 139

\bibitem{Hassan2012} Hassan, S.F., \& Rosen, R.A. 2012, JHEP, 02, 126

\bibitem{SchmidtMay2016} Schmidt-May, A., \& von Strauss, M. 2016, J. Phys. A, 49, 183001

\bibitem{BoylanKolchin2023} Boylan-Kolchin, M. 2023, Nature Astronomy, 7, 731

\bibitem{Tacchella2022} Tacchella, S., et al. 2022, ApJ, 927, 170

\end{thebibliography}

\end{document}
